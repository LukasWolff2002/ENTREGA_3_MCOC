\section*{Resumen}
Una ataguía es una estructura temporal utilizada para drenar zonas cubiertas de agua \textbf{\cite{madanayaka2018}}. En su diseño, es importante considerar factores como el caudal de infiltración, las presiones de poros y la estabilidad, siendo la licuefacción un fenómeno crítico. Este ocurre cuando las tensiones internas del suelo disminuyen, convirtiendo la mezcla de agua y sedimentos en un fluido, lo que puede comprometer la estructura \textbf{\cite{sumer2009}}.
\\ \\
Existen diversos métodos para analizar redes de flujo: el cálculo teórico, el uso de grillas con diferencias finitas y los modelos a escala. El proyecto comparó estos métodos y concluyó que el de diferencias finitas es el más eficaz por su rapidez y precisión, aunque requiere un análisis previo. El cálculo teórico es lento y propenso a errores, mientras que los modelos a escala son útiles para calibrar los resultados.
\\ \\
Finalmente, se utilizó el caudal de infiltración como parámetro de comparación entre los distintos modelos, observándose una diferencia de alrededor del 25\% entre el método teórico y el de diferencias finitas. Posteriormente, utilizando el modelo a escala, se logró calibrar el código de diferencias finitas, obteniendo un error final del 7\%. Considerando todas las fuentes de error discutidas a lo largo del proyecto, este error puede considerarse dentro de un rango esperado.
