\section*{Resumen}
Una ataguía es una estructura temporal utilizada para drenar zonas cubiertas de agua y facilitar trabajos. En su diseño, es importante considerar factores como el caudal de infiltración, presiones de poros y estabilidad, siendo la licuefacción un fenómeno crítico. Este ocurre cuando las tensiones internas del suelo disminuyen, convirtiendo la mezcla de agua y sedimentos en un fluido, lo que puede destruir la estructura.
\\ \\
Existen diversos métodos para analizar redes de flujo: el cálculo teórico usando líneas equipotenciales y de flujo, el uso de grillas con diferencias finitas y modelos a escala. El proyecto comparó estos métodos y concluyó que el de diferencias finitas es el más eficaz por su rapidez y precisión, aunque requiere un análisis previo. El cálculo teórico es lento y propenso a errores, mientras que los modelos a escala son útiles para calibrar los resultados.
\\ \\
Finalmente, se midió el caudal de infiltración, observando un error máximo en los resultados.