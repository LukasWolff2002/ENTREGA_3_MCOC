\part{Anexos}

\section{Repositorio GitHub}

Toda la información de este proyecto se encuentra en el siguiente \href{https://github.com/LukasWolff2002/ENTREGA_3_MCOC}{link}.

\section{Tablas}

\begin{table}[H]
    \centering
    \begin{tabular}{|c|c|c|c|c|c|c|c|}
    \hline
    Caso & $a_1$ & $b_1$ & $c_1$ & $a_2$ & $b_2$ & $c_2$ & $d$ \\ \hline
    1    & 0.8   & 3.8   & 18.6  & 10.0  & 3.0   & 10.2  & 0.0 \\ \hline
    2    & 0.8   & 3.8   & 18.6  & 7.6   & 3.0   & 12.6  & 2.4  \\ \hline
    3    & 0.8   & 3.8   & 18.6  & 6.2   & 1.0   & 16.0  & 5.8  \\ \hline
    \end{tabular}
    \caption{Valores para cada caso [m]}
    \label{tab:medidas}
\end{table}

\begin{table}[H]
    \centering
    \begin{tabular}{|c|c|c|c|c|c|c|c|}
    \hline
    Caso  & $b_1$ & $c_1$  & $b_2$ & $c_2$ & $d$ \\ \hline
    Modelo Escala [cm] & 8  & 20.5  & 0.5   & 17.5  & 5.5 \\ \hline
    Modelo Escala [m] & 6.8  & 17.424 & 0.425   & 14.875  & 4.675  \\ \hline
    \end{tabular}
    \caption{Medidas Escala 1:85}
    \label{tab:medidas_escaladas}
\end{table}

\section{Códigos}

A continuación, se presentan links al repositorio de GitHub que contiene toda la información necesaria para ejecutar los códigos.

\subsection{Códigos de Cálculo Teórico}

Para configurar las variables del modelo, trabajar en el \href{https://github.com/LukasWolff2002/ENTREGA_3_MCOC/blob/main/CODIGO/CALCULOS_MANUALES/variables.py}{siguiente código}. Luego, para ejecutar los cálculos teóricos y obtener los distintos resultados y gráficos, ejecutar el \href{https://github.com/LukasWolff2002/ENTREGA_3_MCOC/blob/main/CODIGO/CALCULOS_MANUALES/main.py}{siguiente código}.

\subsection{Códigos de Diferencias Finitas}

Modificar el \href{https://github.com/LukasWolff2002/ENTREGA_3_MCOC/blob/main/CODIGO/LAPLACE/variables.py}{siguiente código} para determinar las variables del modelo y ejecutar el \href{https://github.com/LukasWolff2002/ENTREGA_3_MCOC/blob/main/CODIGO/LAPLACE/main.py}{siguiente código} para obtener los resultados y gráficos. Todo el cálculo de diferencias finitas es realizado por el \href{https://github.com/LukasWolff2002/ENTREGA_3_MCOC/blob/main/CODIGO/LAPLACE/laplace_solver.py}{siguiente código}. 
