\section*{Introduccion}

Una ataguia es una estructura temporal utilizada para drenar zonas cubiertas de agua, de este modo, es posible construir en terrenos
que de otra forma serian inaccesibles \textbf{\cite{madanayaka2018}}. Hay varios factores importantes a determinar al momento de diseñar
una ataguia, como por ejemplo el caudal de infiltracion, las presiones de poros, estabilidad de la estructura y por sobre todo la licuefaccion
y su factor de seguridad (FS). Este ultimo fenomeno ocurre cuando las presiones de poros alcanzan tal punto, que las tensiones internas efectivas
entre las particulas de suelo pierden efectividad, y en consecuencia, la mescla entre agua y sedimentos actua como un fluido \textbf{\cite{sumer2009}}
\\ \\
El presente proyecto tiene como objetivo el estido y analisis de 3 ataguias distintas, donde se buscara evaluar sus distintas caracteristicas utilizando 
calculos manuales a travez de python, un solver mediante diferencias finitas y una maqueta a escala. De esta forma, se buscara analizar la efectividad
de cada metodo de analisis, ademas, de una comparacion directa entre los resultados obtenidos.
\\ \\
Para los calculos manuales, se utilizo la ley de Darcy, la cual expone 
\\ \\
Hablar de diferencias finitas
\\ \\
El modelo base utilizado a lo largo del informe es el siguiente:
\\ \\
PONER IMAGEEEEEN!!!!
\\ \\
Las medidas para los distintos casos se encuentran en la tabla \textbf{\ref{tab:medidas}}.

