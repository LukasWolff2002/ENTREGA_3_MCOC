\part{Calculos Manuales}

\section{Teoria}
\subsection{Líneas de flujo}
Las líneas de flujo representan los caminos que siguen las partículas de agua mientras se mueven a través de un medio poroso. Estas líneas son perpendiculares a las líneas equipotenciales y a las líneas de corriente. Las líneas de flujo son paralelas a la dirección del flujo de agua subterránea.
\subsection{Líneas Equipotenciales}
Las líneas equipotenciales representan zonas en donde el potencial hidráulico es constante. Esto significa, que no hay diferencia de energía entre dos puntos conectados por una línea equipotencial. Estas líneas sirven para poder visualizar la dirección del flujo de agua subterránea.
\subsection{Presión de Poros}
La presión de poros es la fuerza que el agua ejerce dentro de los poros de un material como el suelo o la roca. En la ingeniería geotécnica, es fundamental porque una presión de poros demasiado alta puede disminuir el esfuerzo efectivo del suelo. Esto puede ocasionar problemas graves, como la licuefacción, donde el suelo pierde su firmeza, o la formación de tuberías subterráneas. En las ataguías, la presión de poros es especialmente importante porque influye directamente en la estabilidad del terreno que rodea la estructura, lo que es clave para evitar fallas.
\subsection{Presión en Ataguía}
\subsubsection{Presión Total}
\subsubsection{Presión Efectiva}
\subsubsection{Estabilidad}
\section{Resultados}

\subsection{Diagramas Escala 1:200}

\begin{figure}[H]
    \centering
    \begin{minipage}{0.32\textwidth}
        \centering
        \includegraphics[width=\textwidth]{GRAFICOS/caso_1.jpg}
        \caption{Caso 1}
    \end{minipage}
    \begin{minipage}{0.32\textwidth}
        \centering
        \includegraphics[width=\textwidth]{GRAFICOS/caso_2.jpg}
        \caption{Caso 2}
    \end{minipage}
    \begin{minipage}{0.32\textwidth}
        \centering
        \includegraphics[width=\textwidth]{GRAFICOS/caso_3.jpg}
        \caption{Caso 3}
    \end{minipage}
  \end{figure}

\subsection{Presion de Poros}

\subsubsection{Distribucion Presiones}

\begin{figure}[H]
    \centering
    \begin{minipage}{0.32\textwidth}
        \centering
        \includegraphics[width=\textwidth]{GRAFICOS/caso_1_presion_poros.jpg}
        \caption{Caso 1 Presion Poros}
    \end{minipage}
    \begin{minipage}{0.32\textwidth}
        \centering
        \includegraphics[width=\textwidth]{GRAFICOS/caso_2_presion_poros.jpg}
        \caption{Caso 2 Presion Poros}
    \end{minipage}
    \begin{minipage}{0.32\textwidth}
        \centering
        \includegraphics[width=\textwidth]{GRAFICOS/caso_3_presion_poros.jpg}
        \caption{Caso 3 Presion Poros}
    \end{minipage}
\end{figure}

\subsubsection{Presiones Totales}

\begin{figure}[H]
    \centering
    \begin{minipage}{0.32\textwidth}
        \centering
        \includegraphics[width=\textwidth]{GRAFICOS/caso_1_presion_ataguia_total.jpg}
        \caption{Caso 1 Presion Ataguia Total}
    \end{minipage}
    \begin{minipage}{0.32\textwidth}
        \centering
        \includegraphics[width=\textwidth]{GRAFICOS/caso_2_presion_ataguia_total.jpg}
        \caption{Caso 2 Presion Ataguia Total}
    \end{minipage}
    \begin{minipage}{0.32\textwidth}
        \centering
        \includegraphics[width=\textwidth]{GRAFICOS/caso_3_presion_ataguia_total.jpg}
        \caption{Caso 3 Presion Ataguia Total}
    \end{minipage}
\end{figure}

\subsubsection{Presiones Efectivas}

\begin{figure}[H]
    \centering
    \begin{minipage}{0.32\textwidth}
        \centering
        \includegraphics[width=\textwidth]{GRAFICOS/caso_1_presion_ataguia_neta.jpg}
        \caption{Caso 1 Presion Ataguia Neta}
    \end{minipage}
    \begin{minipage}{0.32\textwidth}
        \centering
        \includegraphics[width=\textwidth]{GRAFICOS/caso_2_presion_ataguia_neta.jpg}
        \caption{Caso 2 Presion Ataguia Neta}
    \end{minipage}
    \begin{minipage}{0.32\textwidth}
        \centering
        \includegraphics[width=\textwidth]{GRAFICOS/caso_3_presion_ataguia_neta.jpg}
        \caption{Caso 3 Presion Ataguia Neta}
    \end{minipage}
\end{figure}

\subsubsection{Estabilidad}

\begin{figure}[H]
    \centering
    \begin{minipage}{0.32\textwidth}
        \centering
        \includegraphics[width=\textwidth]{GRAFICOS/caso_1_centroide_y.jpg}
        \caption{Caso 1 Centroide}
    \end{minipage}
    \begin{minipage}{0.32\textwidth}
        \centering
        \includegraphics[width=\textwidth]{GRAFICOS/caso_2_centroide_y.jpg}
        \caption{Caso 2 Centroide}
    \end{minipage}
    \begin{minipage}{0.32\textwidth}
        \centering
        \includegraphics[width=\textwidth]{GRAFICOS/caso_3_centroide_y.jpg}
        \caption{Caso 3 Centroide}
    \end{minipage}
\end{figure}


