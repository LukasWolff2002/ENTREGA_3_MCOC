\part{Modelo Escala}

\section{Resultados}

\subsection{Calculo Permeabilidad Muestra}

\subsection{Aplicacion Diferencias Finitas}

Se determina un caudal de 0.008972885614821659 m/s

\subsection{Licuefaccion}

A continuacion se presenta un video (ver en Adobe Acrobat) de la falla observada por licuefaccion en la maqueta a escala.

\begin{center}
    \includemedia[
        width=0.5625\textwidth, % Relación de aspecto 9:16 (altura mayor que el ancho)
        height=\textwidth,
        activate=onclick,
        addresource=VIDEOS/licuefaccion.mp4,
        flashvars={
            source=VIDEOS/licuefaccion.mp4
        }
    ]{\includegraphics[width=\textwidth]{VIDEOS/miniatura_licuefaccion.png}}{VPlayer.swf}
\end{center}

Las medidas registradas son las siguientes:

\begin{table}[H]
    \centering
    \begin{tabular}{|c|c|c|c|c|c|c|c|}
    \hline
    Caso & $a_1$ & $b_1$ & $c_1$ & $a_2$ & $b_2$ & $c_2$ & $d$ \\ \hline
    Licuefaccion    & 0.0   & 14.5   & 15.5  & 15  & 2.5   & 12.5  & 0.5 \\ \hline
    \end{tabular}
    \caption{Medidas para la Licuefaccion [cm]}
    \label{tab:medidas1}
\end{table}
    

