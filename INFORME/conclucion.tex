\part{Conclusión}

En conclusión, el presente proyecto tuvo como objetivo el estudio y análisis de 3 ataguías distintas, evaluando sus características mediante cálculos manuales utilizando Python, un solver de diferencias finitas y una maqueta a escala. Se buscó analizar la efectividad de cada método de análisis, así como realizar una comparación directa entre los resultados obtenidos. 
\\ \\
Analizando los 3 casos de estudio, se determinó que el caso 3 presenta la mayor estabilidad, tanto por la licuefacción como por el momento generado por la presión de poros sobre la estructura (aunque solo se analizó una sección de toda la ataguía). Se observó que el caudal de infiltración está directamente relacionado con el gradiente hidráulico, lo que corrobora la Ley de Darcy. Finalmente, mediante un mapa de calor, se visualizó la distribución de las presiones de poros, además de la concentración puntual que ocurre bajo la ataguía cuando se produce la licuefacción.
\\ \\
En cuanto a los cálculos teóricos, se logró generar a través de Python un cálculo correcto. Al comparar con las otras formas de resolución, se concluye que la resolución teórica es bastante confiable, aunque es difícil alcanzar una gran precisión debido a la complejidad que esto implica. Las variaciones en los resultados y el porcentaje de error se encuentran dentro de un rango esperado.
\\ \\
Aplicando diferencias finitas, se concluye que la precisión de los resultados puede ser mucho mayor. Sin embargo, llega un punto en el que aumentar la densidad de la grilla deja de ser relevante, ya que el error se mantiene prácticamente constante. Este modelo es más complejo de aplicar, ya que requiere una comprensión previa de la teoría, pero es mucho más eficiente y podría llegar a competir con programas de pago actuales. Se observa que una cuadrícula de 40x40 para las dimensiones utilizadas es suficiente. Por el contrario, aumentar el tamaño de esta no mejoraría significativamente la precisión de los resultados.
\\ \\
Comparando ambos métodos con el modelo a escala, se concluye que el cálculo mediante diferencias finitas es mucho más preciso, con un error del 7\% en relación con el modelo a escala. Por otro lado, en la comparación entre modelos teóricos y numéricos, se observa un error en torno al 25\%. Reducir este error es muy complejo en la parte teórica, ya que requeriría aumentar el número de canales y líneas equipotenciales, lo que aumenta considerablemente la complejidad del cálculo.
\\ \\
Finalmente, el modelo a escala fue fundamental en la realización de este proyecto, ya que no solo permitió observar los distintos fenómenos estudiados, sino que también facilitó la calibración y corroboración de la información proporcionada por los modelos teóricos y computacionales. Sin embargo, se critica el tipo de suelo utilizado, ya que su alta permeabilidad dificulta el control de los valores obtenidos. Además, es un tipo de suelo que difícilmente se encontraría en una estructura como esta.
