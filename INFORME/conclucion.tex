\section{Conclucion}

%Párrafo 1. Resumen de lo que se hizo en el proyecto 
%Párrafo 2. Resumen de los modelos utilizados
%Párrafo 3 a 5 (aprox.) discusión de comparación de resultados entre las distintas aproximaciones (numérica y experimental) y resultados estadísticos. 
%Párrafo final. Crítica al modelo (supuestos, metodología, caso real vs caso simulado, etc.)

En conclucion, el presente proyecto tuvo como objetivo el estudio y análisis de 3 ataguías distintas, donde se buscó evaluar sus distintas características utilizando cálculos manuales a través de python, un solver mediante diferencias finitas y una maqueta a escala. 
De esta forma, se buscó analizar la efectividad de cada método de análisis, además de una comparación directa entre los resultados obtenidos. 
\\ \\
Analizando los 3 casos de estudio, se determino que el caso 3 es aque que presenta una mayor estabilidad, tanto por la licuefaccion, como por el momento que genera la presion de poros sobre la estructura misma (aun asi solo se esta analizando una seccion de toda la ataguia).
Se observo que el caudal de infiltracion esta directamente relacionado con el gradiente hidraulico, corrobroando la ley de Darcy. Finalmente, mediante un mapa de calor, se logro observar como se distribuyen las presiones de poros, ademas de la concentracion puntual que se da bajo
la ataguia cuando se produce la licuefaccion.
\\ \\
En cuanto a los calculos teoricos
\\ \\
Aplicando diferencias finitas
\\ \\
Comparando ambos metodos con el modelo a escala, se concluye que el calculo mediante diferencias finitas es mucho mas exacto, observando un error del 
\\ \\
Criticar modelo