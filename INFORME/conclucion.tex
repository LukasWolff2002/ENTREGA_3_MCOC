\part{Conclucion}

%Párrafo 1. Resumen de lo que se hizo en el proyecto 
%Párrafo 2. Resumen de los modelos utilizados
%Párrafo 3 a 5 (aprox.) discusión de comparación de resultados entre las distintas aproximaciones (numérica y experimental) y resultados estadísticos. 
%Párrafo final. Crítica al modelo (supuestos, metodología, caso real vs caso simulado, etc.)

En conclucion, el presente proyecto tuvo como objetivo el estudio y análisis de 3 ataguías distintas, donde se buscó evaluar sus distintas características utilizando cálculos manuales a través de python, un solver mediante diferencias finitas y una maqueta a escala. 
De esta forma, se buscó analizar la efectividad de cada método de análisis, además de una comparación directa entre los resultados obtenidos. 
\\ \\
Analizando los 3 casos de estudio, se determino que el caso 3 es aque que presenta una mayor estabilidad, tanto por la licuefaccion, como por el momento que genera la presion de poros sobre la estructura misma (aun asi solo se esta analizando una seccion de toda la ataguia).
Se observo que el caudal de infiltracion esta directamente relacionado con el gradiente hidraulico, corrobroando la ley de Darcy. Finalmente, mediante un mapa de calor, se logro observar como se distribuyen las presiones de poros, ademas de la concentracion puntual que se da bajo
la ataguia cuando se produce la licuefaccion.
\\ \\
En cuanto a los calculos teoricos, se logro generar a travez de python un calculo correcto, donde posteriormente al comparar con las otras froams de resolucion, se puede concluir que la resolucion teorica es bastante fiable, aun asi, es dificil alcanzar una gran precision debido a la complejidad que esto conlleva. 
Las variaciones en los resultados, ademas del porcentaje de error estan dentro de un rango esperado.
\\ \\
Aplicando diferencias finitas se concluye que la precision de los resultados puede ser mucho mayor, aun asi, existe un punto donde aumentar la grilla deja de ser relevante, ya que el error se mantiene practicamente constante. Este modelo es mas complejo de aplicar, ya que rerquiere una comprencion de la teoria previamente, aun asi, es mucho mas eficiente y podria llegar a competir con los programas de pago actuales.
\\ \\
Comparando ambos metodos con el modelo a escala, se concluye que el calculo mediante diferencias finitas es mucho mas exacto, observando un error del 7\% con el modelo a escala realizado. Por otro lado, y comparando entre modelos, se observa un error en torno al 25\%, donde disminuir este dato es muy complejo por la parte torica, ya que requiere aumentar el numero de canales y lineas equipotenciales, lo cual aumenta la complejidad del calculo.
\\ \\
Finalmente el modelo a escala fue muy importante el la realizacion de esgte proyecto, ya que no solo permitio observar los distitnos fenomenos estudiados, sino que tambien pewrmitio calibrar y corroborar la informacion entregada por los modelos teoricos y computacionales. Aun asi, se critica el tipo de suelo utilizado, ya que este al tener una permeabilidad tan alta, genera que los valores obtenidos sean dificiles de controlar, donde ademas, es un suelo que dificilmente se presentara en una estructura como esta.
\\ \\
