\part{Diferencias Finitas}

Para mejorar la precisión de los resultados, se implementó un modelo de diferencias finitas aplicado a la matriz de potencial hidráulico. De esta manera, se pueden calcular las líneas de flujo, obteniendo así el caudal de infiltración. El objetivo principal es determinar la efectividad de este método frente a un proceso más simple, como el cálculo manual expuesto en la sección anterior.

\section{Teoría}

\subsection{Ley de Darcy}

La ley de Darcy se expresa como:

\begin{equation}
    q = k \cdot i \cdot A
\end{equation}

Lo cual es análogo a:

\begin{equation}
    v = k \cdot i
\end{equation}

Donde \(i\) es el gradiente hidráulico. Discretizando en el espacio, se obtiene lo siguiente:

\begin{equation}
    i = \frac{dh}{dl} = \frac{dh}{dx}; \frac{dh}{dy}; \frac{dh}{dz}
\end{equation}

Sea lo siguiente:

\begin{figure}[H]
    \centering
    \includegraphics[width=0.5\textwidth]{FOTOS/in.jpg}
    \caption{Entrada al sistema}
    \label{fig:ley_darcy_in}
\end{figure}

La serie de Taylor expone que:

\begin{equation}
    f(x) = f(a) + \frac{df(a) \Delta X}{dx \cdot 1!} + ... + \frac{\Delta X^n}{n!} \cdot \frac{d^n f(a)}{dx^n}
\end{equation}

Por lo tanto, lo que sale del sistema es:

\begin{figure}[H]
    \centering
    \includegraphics[width=0.5\textwidth]{FOTOS/out.jpg}
    \caption{Salida del sistema}
    \label{fig:ley_darcy_out}
\end{figure}

Luego, por conservación de masa:

\begin{equation}
    Q_{int} = Q_{out}
\end{equation}

De lo que se obtiene:

\begin{equation}
    \rho_u \Delta_y \Delta_z + \rho_v \Delta_x \Delta_z + \rho_w \Delta_x \Delta_y = (\rho_u + \frac{d\rho_u \Delta x}{dx})\Delta_y \Delta_z + (\rho_v + \frac{d\rho_v \Delta y}{dy})\Delta_x \Delta_z + (\rho_w + \frac{d\rho_w \Delta z}{dz})\Delta_x \Delta_y
\end{equation}

Simplificando:

\begin{equation}
    \Delta_x \Delta_y \Delta_z = Volumen
\end{equation}

Dado que el fluido es agua, que es incompresible, se tiene:

\begin{equation}
   -\rho(\frac{du}{dx}+ \frac{dv}{dy}+ \frac{dw}{dz}) = 0
\end{equation}

Lo cual es análogo a:

\begin{equation}
    -\rho \nabla \cdot \vec{v} = 0 = \nabla \cdot \vec{v}
\end{equation}

Por lo tanto, al reemplazar en la ley de Darcy, se obtiene:

\begin{equation}
    V_x = k_x\cdot \frac{dh}{dx}; V_y = k_y\cdot \frac{dh}{dy}; V_z = k_z\cdot \frac{dh}{dz}
\end{equation}

Incorporando la ecuación de continuidad, se obtiene:

\begin{equation}
    \nabla \cdot \vec{V} = \nabla \cdot (k \cdot \vec{i}) = 0
\end{equation}

Asumiendo un análisis en 2D, se tiene:

\begin{equation}
    \frac{d}{dx}(k_x \cdot \frac{dh}{dx}) + \frac{d}{dy}(k_y \cdot \frac{dh}{dy}) = 0
\end{equation}

Suponiendo que:

\begin{equation}
    k_x = k_y = k
\end{equation}

Se obtiene:

\begin{equation}
    k \nabla^2 h = 0
\end{equation}

De esta forma, podemos representar el laplaciano mediante diferencias finitas.

\subsection{Diferencias Finitas}

\subsubsection{Diferencias Hacia Adelante}

\begin{equation}
    h(x + \Delta x) = h(x) + \frac{dh}{dx} \Delta x + ...
\end{equation}

\subsubsection{Diferencias Hacia Atrás}

\begin{equation}
    h(x - \Delta x) = h(x) - \frac{dh}{dx} \Delta x + ...
\end{equation}

\subsubsection{Diferencias Centrales}

La suma de las diferencias hacia adelante y hacia atrás es:

\begin{equation}
    h(x + \Delta x) + h(x - \Delta x) = h(x) + \frac{d^2h}{dx^2}\frac{\Delta x}{2!} + ...(los pares)
\end{equation}

Donde la incógnita es $\frac{d^2h}{dx^2}$. Despejando, obtenemos:

\begin{equation}
    \frac{d^2h}{dx^2} = \frac{h(x + \Delta x) - 2h(x) + h(x - \Delta x)}{\Delta x^2}
\end{equation}

\begin{equation}
    \frac{dh}{dx} = \frac{h(x + \Delta x) - h(x)}{\Delta x}
\end{equation}

Esto se puede llevar a una grilla:

\begin{figure}[H]
    \centering
    \includegraphics[width=0.5\textwidth]{FOTOS/grilla.jpg}
    \caption{Grilla}
\end{figure}

Donde se puede representar la ecuación de Laplace como:

\begin{equation}
    \frac{d^2h}{dx^2} = \frac{h_{i+1,j} + h_{i-1,j} - 2h_{i,j}}{\Delta x^2}
\end{equation}

\begin{equation}
    \frac{dh}{dx} = \frac{h_{+1,j} + h_{i+1,j}}{2\Delta x}
\end{equation}

Por lo tanto, podemos expresar la ley de Darcy con diferencias centrales, obteniendo:

\begin{equation}
    \frac{k}{\Delta^2}(h_{i+1,j} + h_{i-1,j} + h_{i,j+1} + h_{i,j-1} - 4h_{i,j}) = 0
\end{equation}

Donde se busca:

\begin{equation}
    h_{i,j} = \frac{1}{4}(h_{i+1,j} + h_{i-1,j} + h_{i,j+1} + h_{i,j-1})
\end{equation}

De esta forma, es posible obtener las variaciones en el potencial a partir de los datos conocidos en la grilla (condiciones de borde).

\newpage

\section{Resultados usando Diferencias Finitas}
Luego de implementar el método de diferencias finitas, se obtuvieron los siguientes resultados para los casos planteados en la sección anterior.
\subsection{Caso 1}

\begin{figure}[H]
    \centering
    \includegraphics[width=\textwidth]{GRAFICOS/laplace_caso_1.jpg}
    \caption{Caso 1 Laplace}
    \label{fig:laplace_caso_1}
\end{figure}

Como se puede apreciar en la figura \ref{fig:laplace_caso_1}, tiene el mayor potencial hidráulico abajo de la ataguía, lo cual conlleva a un caudal de infiltración mayor que en los otros casos.

\subsection{Caso 2}

\begin{figure}[H]
    \centering
    \includegraphics[width=\textwidth]{GRAFICOS/laplace_caso_2.jpg}
    \caption{Caso 2 Laplace}
    \label{fig:laplace_caso_2}
\end{figure}

Para el caso 2, se observa que se ocupó el mismo modelo, pero se cambiaron las dimensiones y potenciales hidráulicos. Este cambio, se puede observar en la distribución de la matriz K y en la matriz de potenciales. Además, en este caso el potencial hidráulico abajo de la ataguía es menor.

\subsection{Caso 3}

\begin{figure}[H]
    \centering
    \includegraphics[width=\textwidth]{GRAFICOS/laplace_caso_3.jpg}
    \caption{Caso 3 Laplace}
    \label{fig:laplace_caso_3}
\end{figure}

Finalmente, en la figura \ref{fig:laplace_caso_3} se cambiaron la geometría del problema, distribución de la matriz k y los potenciales hidráulicos. En este caso, el potencial hidráulico abajo de la ataguía es el menor de los tres casos, lo cual conlleva a un caudal de infiltración menor.

\section{Análisis de Resultados}

\begin{table}[H]
    \centering
    \caption{Caudales obtenidos mediante análisis analítico y diferencias finitas.}
    \vspace{0.5cm}
    \begin{tabular}{cccc}
        \hline
        \textbf{Caso} & \textbf{Caudal de Infiltración [$m/dia$]} & \textbf{Caudal según Laplace [$m/dia$]} & \textbf{Error [\%]} \\
        \hline
        1 & $0.97$ & $1.21$ & $24.7$ \\
        2 & $0.72$ & $0.99$ & $36.0$ \\
        3 & $0.57$ & $0.66$ & $15.8$ \\
        \hline
    \end{tabular}
    \label{tab:Diferencias1}
\end{table}

Los resultados comparan el caudal de infiltración obtenido de manera analítica con el caudal calculado por el método de diferencias finitas. En el Caso 1, el error es del 24.7\%, en el Caso 2 es mayor, con un 36.0\%, y en el Caso 3 es menor, con un 15.8\%. A pesar de los errores, ambos métodos coinciden en que el Caso 1 presenta el mayor caudal de infiltración y el Caso 3 con el menor caudal de infiltración. Esto sugiere que, aunque ambos métodos son similares, el método de diferencias finitas puede tener desviaciones dependiendo de la cantidad de nodos, mientras que la manera analítica depende de la cantidad de equipotenciales y líneas de flujo. Estas condiciones del sistema afectan los cálculos, que se pueden ver en el margen de error.


El método de diferencias finitas es una opción de cálculo más precisa que el método manual, ya que incorpora cada parte de la ataguía infinitesimalmente, calculando las presiones de manera más homogénea. Como resultado, se obtuvieron caudales menores y más representativos para cada caso, siendo los casos 1 y 2 caudales relativamente cercanos debido a que la altura de agua en ambos lados de la ataguía es la misma.
