\part{Diferencias Finitas}

\section{Teoria}

\subsection{Ley de Darcy}

La ley de Darcy expone lo siguiente:

\begin{equation}
    q = k \cdot i \cdot A
\end{equation}

Lo cual es análogo a:

\begin{equation}
    v = k \cdot i
\end{equation}

Donde \(i\) es el gradiente hidráulico. Discretizándolo en el espacio, se obtiene lo siguiente:

\begin{equation}
    i = \frac{dh}{dl} = \frac{dh}{dx}; \frac{dh}{dy}; \frac{dh}{dz}
\end{equation}

Sea lo siguiente:

\begin{figure}[H]
    \centering
    \includegraphics[width=0.5\textwidth]{FOTOS/in.jpg}
    \caption{Entrada al sistema}
    \label{fig:ley_darcy_in}
\end{figure}

La serie de Taylor expone que:

\begin{equation}
    f(x) = f(a) + \frac{df(a) \Delta X}{dx \cdot 1!} + ... + \frac{\Delta X^n}{n!} \cdot \frac{d^n f(a)}{dx^n}
\end{equation}

Por lo tanto, lo que sale del sistema es:

\begin{figure}[H]
    \centering
    \includegraphics[width=0.5\textwidth]{FOTOS/out.jpg}
    \caption{Salida del sistema}
    \label{fig:ley_darcy_out}
\end{figure}

Luego, por conservación de masa:

\begin{equation}
    Q_{int} = Q_{out}
\end{equation}

De lo que se obtiene:

\begin{equation}
    \rho_u \Delta_y \Delta_z + \rho_v \Delta_x \Delta_z + \rho_w \Delta_x \Delta_y = (\rho_u + \frac{d\rho_u \Delta x}{dx})\Delta_y \Delta_z + (\rho_v + \frac{d\rho_v \Delta y}{dy})\Delta_x \Delta_z + (\rho_w + \frac{d\rho_w \Delta z}{dz})\Delta_x \Delta_y
\end{equation}

Simplificando:

\begin{equation}
    \Delta_x \Delta_y \Delta_z = Volumen
\end{equation}

Pero el fluido, al ser agua, es incompresible, por lo tanto:

\begin{equation}
   -\rho(\frac{du}{dx}+ \frac{dv}{dy}+ \frac{dw}{dz}) = 0
\end{equation}

Lo que es análogo a decir:

\begin{equation}
    -\rho \nabla \cdot \vec{v} = 0 = \nabla \cdot \vec{v}
\end{equation}

Por lo tanto, si reemplazamos en la ley de Darcy, obtenemos:

\begin{equation}
    V_x = k_x\cdot \frac{dh}{dx}; V_y = k_y\cdot \frac{dh}{dy}; V_z = k_z\cdot \frac{dh}{dz}
\end{equation}

Incorporando la ecuación de continuidad, se obtiene:

\begin{equation}
    \nabla \cdot \vec{V} = \nabla \cdot (k \cdot \vec{i}) = 0
\end{equation}

Asumiendo un análisis en 2D, se obtiene:

\begin{equation}
    \frac{d}{dx}(k_x \cdot \frac{dh}{dx}) + \frac{d}{dy}(k_y \cdot \frac{dh}{dy}) = 0
\end{equation}

Pero sabemos, o mejor dicho, suponemos que:

\begin{equation}
    k_x = k_y = k
\end{equation}

Por lo tanto:

\begin{equation}
    k \nabla^2 h = 0
\end{equation}

De esta forma, podemos representar el laplaciano con diferencias finitas.

\subsection{Diferencias Finitas}

\subsubsection{Diferencias Hacia Adelante}

\begin{equation}
    h(x + \Delta x) = h(x) + \frac{dh}{dx} \Delta x + ...
\end{equation}

\subsubsection{Diferencias Hacia Atrás}

\begin{equation}
    h(x - \Delta x) = h(x) - \frac{dh}{dx} \Delta x + ...-...+
\end{equation}

\subsubsection{Diferencias Centrales}

Se representa como la suma de una diferencia hacia adelante y hacia atrás, obteniendo:

\begin{equation}
    h(x + \Delta x) + h(x - \Delta x) = h(x) + \frac{d^2h}{dx^2}\frac{\Delta x}{2!} + ...(los pares)
\end{equation}

Donde la incógnita que se busca es $\frac{d^2h}{dx^2}$, por lo tanto, despejando, se obtiene:

\begin{equation}
    \frac{d^2h}{dx^2} = \frac{h(x + \Delta x) - 2h(x) + h(x - \Delta x)}{\Delta x^2}
\end{equation}

\begin{equation}
    \frac{dh}{dx} = \frac{h(x + \Delta x) - h(x)}{\Delta x}
\end{equation}

Lo cual se puede llevar a una grilla:

\begin{figure}[H]
    \centering
    \includegraphics[width=0.5\textwidth]{FOTOS/grilla.jpg}
    \caption{Grilla}
\end{figure}

Donde se puede representar la ecuación de Laplace como:

\begin{equation}
    \frac{d^2h}{dx^2} = \frac{h_{i+1,j} + h_{i-1,j} - 2h_{i,j}}{\Delta x^2}
\end{equation}

\begin{equation}
    \frac{dh}{dx} = \frac{h_{+1,j} + h_{i+1,j}}{2\Delta x}
\end{equation}

Por lo tanto, podemos expresar la ley de Darcy con diferencias centrales, obteniendo:

\begin{equation}
    \frac{k}{\Delta^2}(h_{i+1,j} + h_{i-1,j} + h_{i,j+1} + h_{i,j-1} - 4h_{i,j}) = 0
\end{equation}

Donde se busca:

\begin{equation}
    h_{i,j} = \frac{1}{4}(h_{i+1,j} + h_{i-1,j} + h_{i,j+1} + h_{i,j-1})
\end{equation}

De esta forma, es posible obtener las diferentes variaciones en el potencial, a partir de los datos conocidos en la grilla (condiciones de borde).

\section{Resultados}

\subsection{Caso 1}

\begin{figure}[H]
    \centering
    \includegraphics[width=\textwidth]{GRAFICOS/laplace_caso_1.jpg}
    \caption{Caso 1 Laplace}
\end{figure}

\subsection{Caso 2}

\begin{figure}[H]
    \centering
    \includegraphics[width=\textwidth]{GRAFICOS/laplace_caso_2.jpg}
    \caption{Caso 2 Laplace}
\end{figure}

\subsection{Caso 3}

\begin{figure}[H]
    \centering
    \includegraphics[width=\textwidth]{GRAFICOS/laplace_caso_3.jpg}
    \caption{Caso 3 Laplace}
\end{figure}
   